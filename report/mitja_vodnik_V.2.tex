\documentclass[a4paper]{article}
\usepackage[slovene]{babel}
\usepackage[utf8]{inputenc}
\usepackage[T1]{fontenc}
%\usepackage[margin=2cm, bottom=3cm, foot=1.5cm]{geometry}
\usepackage{float}
\usepackage{graphicx}
\usepackage{amsmath}
\usepackage{amssymb}
\usepackage{subcaption}
\usepackage{hyperref}
\usepackage{dirtytalk}

\newcommand{\tht}{\theta}
\newcommand{\Tht}{\Theta}
\newcommand{\dlt}{\delta}
\newcommand{\eps}{\epsilon}
\newcommand{\thalf}{\frac{3}{2}}
\newcommand{\ddx}[1]{\frac{d^2#1}{dx^2}}
\newcommand{\ddr}[2]{\frac{\partial^2#1}{\partial#2^2}}
\newcommand{\mddr}[3]{\frac{\partial^2#1}{\partial#2\partial#3}}

\newcommand{\der}[2]{\frac{d#1}{d#2}}
\newcommand{\pder}[2]{\frac{\partial#1}{\partial#2}}
\newcommand{\half}{\frac{1}{2}}
\newcommand{\forth}{\frac{1}{4}}
\newcommand{\q}{\underline{q}}
\newcommand{\p}{\underline{p}}
\newcommand{\x}{\underline{x}}
\newcommand{\liu}{\hat{\mathcal{L}}}
\newcommand{\bigO}[1]{\mathcal{O}\left( #1 \right)}
\newcommand{\pauli}{\mathbf{\sigma}}
\newcommand{\bra}[1]{\langle#1|}
\newcommand{\ket}[1]{|#1\rangle}
\newcommand{\id}[1]{\mathbf{1}_{2^{#1}}}
\newcommand{\tinv}{\frac{1}{\tau}}
\newcommand{\s}{\sigma}
\newcommand{\vs}{\vec{\s}}
\newcommand{\vr}{\vec{r}}
\newcommand{\vq}{\vec{q}}
\newcommand{\vv}{\vec{v}}
\newcommand{\vo}{\vec{\omega}}
\newcommand{\uvs}{\underline{\vs}}
\newcommand{\expected}[1]{\left\langle #1 \right\rangle}
\newcommand{\D}{\Delta}

\newcommand{\range}[2]{#1, \ldots, #2}
\newcommand{\seq}[2]{#1 \ldots #2}
\newcommand{\psiCoef}[2]{\psi_{\range{#1}{#2}}}
\newcommand{\psiCoeff}[3]{\psi_{#1, \range{#2}{#3}}}
\newcommand{\mpa}[2]{#1^{(#2)}_{s_#2}}
\newcommand{\us}{\underline{s}}
\newcommand{\up}{\uparrow}
\newcommand{\down}{\downarrow}

\begin{document}

    \title{\sc\large Višje računske metode\\
		\bigskip
		\bf\Large DMRG metode}
	\author{Mitja Vodnik, 28182041}
            \date{\today}
	\maketitle

    V tej nalogi nas zanima osnovno stanje Heisenbergove verige $n$ antiferomagnetnih kvantnih spinov:

    \begin{equation}\label{eq1}
        H = \sum_{j=1}^{n-1} \vs_j \cdot \vs_{j+1}
    \end{equation}

    Konkretno nas zanimajo energije $E_0$ in spin-spin korelacije $C(j, k) = \expected{\s^z_j\s^z_k}$ v osnovnem stanju.

    \section{TEBD algoritem}

    Naša metoda bo poiskala matrično produktni nastavek (MPA) osnovnega stanja verige. Ideja je naslednja: za začetni približek vzamemo Neelovo
    stanje ($\ket{\psi(0)} = \ket{\down, \up, \ldots, \down, \up}$ = $\ket{1, 0, \ldots, 1, 0}$) in ga \say{ohladimo} do pravega osnovnega
    stanja.
    Ohlajanje dosežemo s propagacijo v imaginarnem času:

    \begin{equation}\label{eq2}
        \ket{\psi(\beta)} = e^{-\beta H} \ket{\psi(0)} \xrightarrow[\beta \to \infty]{} \ket{\psi_0}
    \end{equation}

    Propagacijo izvajamo s pomočjo že znane Trotterjeve formule, s tem da imamo sedaj kvantno stanje zapisano v obliki MPA. TEBD algoritem nam opiše
    delovanje posameznega lokalnega člena propagatorja $U_{j, j+1} = e^{-\beta \vs_j \cdot \vs_{j+1}}$: vsakič je potreben po en singularni razcep,
    ki nam spremeni dve sosednji matriki v MPA, pri čemer se ohrani kanoničnost.

    \section{Energija osnovnega stanja}

    Do energije osnovnega stanja pridemo po dovolj dolgem \say{ohlajanju} začetnega Neelovega stanja $\ket{\psi(0)}$.
    Računamo jo po naslednjem izrazu:

    \begin{equation}\label{eq3}
        E_0 = - \lim_{\beta \to \infty} \frac{1}{\beta} \log{\frac{\bra{\psi(0)}e^{-\beta H}\ket{\psi(0)}}{\langle\psi(0)\ket{\psi(0)}}}
    \end{equation}

    Na sliki~\ref{slika1} za nekaj različnih dolžin verige vidimo, da se po dovolj dolgi propagaciji energija res začne ustaljevati pri neki
    minimalni vrednosti. Vidimo celo, da se z večanjem dolžine verige energija osnovnega stanja na delec bliža neki fiksni vrednosti - limita bi bila
    energija na delec v osnovnem stanju neskončne verige.\\
    Na sliki~\ref{slika2} je prikazano lepo ujemanje limitne vrednosti energije pri propagaciji z energijami osnovnih stanj kratkih verig, dobljenih
    z diagonalizacijo Hamiltonke.
    Algoritem torej uspešno najde osnovno stanje, poleg tega pa je uporaben za precej daljše verige, kot metode, ki smo jih spoznali doslej. 

    \begin{figure}
        \centering
        \includegraphics[width = \textwidth]{slika1.pdf}
        \caption{Energije verige preračunane na delec. Najnižja vrednost, do katere pridemo na grafu je $E_0/n \approx -1.767$}
        \label{slika1}
    \end{figure}

    \begin{figure}
        \centering
        \includegraphics[width = \textwidth]{slika2.pdf}
        \caption{Črne črtkane črte prikazujejo vrednost energije osnovnega stanja dobljeno z diagonalizacijo Hamiltonke. Vidimo, da 
        energije dobljene s TEBD algoritmom konvergirajo točno proti istim vrednostim.}
        \label{slika2}
    \end{figure}

    \section{Spin-spin korelacije v osnovnem stanju}

    Ko s TEBD algoritmom smo uspešno prišli do MPA predstavitve osnovnega stanja verige. Z usteznimi kontrakcijami lahko računamo spin-spin
    korelacije v tem stanju:

    \begin{equation}\label{eq4}
        \begin{split}
            C(j, k) &= \bra{\psi_0} \s^z_j\s^z_k \ket{\psi_0} \\
            &= \bra{L} T^{(2)} \ldots T^{(j-1)}V^{(j)}T^{(j+1)} \ldots T^{(k-1)}V^{(k)}T^{(k+1)} \ldots T^{(n-1)} \ket{R},
        \end{split}
    \end{equation}

    kjer smo označili:

    \begin{equation}\label{eq5}
        T^{(j)} = \sum_{s \in \{0, 1\}} A_s^{(j)} \otimes A_s^{(j)}, \quad
        \bra{L} = T^{(1)}, \quad
        \ket{R} = T^{(n)}
    \end{equation}

    \begin{equation}\label{eq6}
        V^{(j)} = \sum_{s, s' \in \{0, 1\}} (\s^z_j)_{ss'}  A_s^{(j)} \otimes A_{s'}^{(j)}
    \end{equation}

    Tako izračunane korelacije lahko uredimo v matriko - dva primera sta prikazana na sliki~\ref{slika3}.

    \begin{figure}
        \centering
        \begin{subfigure}{\textwidth}
            \includegraphics[width = \textwidth]{slika3a.pdf}
        \end{subfigure}
        \begin{subfigure}{\textwidth}
            \includegraphics[width = \textwidth]{slika3b.pdf}
        \end{subfigure}
        \caption{Prikaza matrik spinskih korelacij v osnovnem stanju dveh različno dolgih verig.}
        \label{slika3}
    \end{figure}

    Postavimo se sedaj na prvi spin in poglejmo, kako spinska korelacija pada z razdaljo. Ta rezultat je prikazan na sliki~\ref{slika4} - korelacija
    eksponentno pade proti nič.\\
    Poglejmo še, ali je spinska korelacija odvisna le od razdalje med spinoma: $C(j, k) \stackrel{?}{=} C(|j - k|)$. Iz naslednjih slik je razvidno,
    da temu ni tako. Slika~\ref{slika5} je dobljena na naslednji način: za vsako medspinsko razdaljo $|j - k|$ je korelacija računana med vsemi
    možnimi ustreznimi mesti (rezultati za dva primera medspinskih razdalj so prikazani na sliki~\ref{slika6}).
    Na grafu so prikazana relativna odstopanja od povprečja izračunanih korelacij. Ta odstopanja so daleč od zanemarljivih,
    torej je korelacija precej odvisna tudi od mesta $j$, s katerega jo računamo in ne le od razdalje.

    \begin{figure}
        \centering
        \includegraphics[width = \textwidth]{slika4.pdf}
        \caption{Padanje spinske korelacije z oddaljevanjem od prvega spina.}
        \label{slika4}
    \end{figure}

    \begin{figure}
        \centering
        \includegraphics[width = \textwidth]{slika5.pdf}
        \caption{Odstopanja korelacij od translacijsko invariantnih vrednosti pokažejo, da korelacije niso odvisne le od medspinskih razdalj.}
        \label{slika5}
    \end{figure}

    \begin{figure}
        \centering
        \begin{subfigure}{\textwidth}
            \includegraphics[width = \textwidth]{slika6a.pdf}
        \end{subfigure}
        \begin{subfigure}{\textwidth}
            \includegraphics[width = \textwidth]{slika6b.pdf}
        \end{subfigure}
        \caption{Primera variacije vrednosti spinskih korelacij pri dveh fiksnih medspinskih razdaljah. ($|j - k| \in \{50, 51\}$)}
        \label{slika6}
    \end{figure}

\end{document}
