\documentclass[a4paper]{article}
\usepackage[slovene]{babel}
\usepackage[utf8]{inputenc}
\usepackage[T1]{fontenc}
%\usepackage[margin=2cm, bottom=3cm, foot=1.5cm]{geometry}
\usepackage{float}
\usepackage{graphicx}
\usepackage{amsmath}
\usepackage{amssymb}
\usepackage{subcaption}
\usepackage{hyperref}
\usepackage{dirtytalk}

\newcommand{\tht}{\theta}
\newcommand{\Tht}{\Theta}
\newcommand{\dlt}{\delta}
\newcommand{\eps}{\epsilon}
\newcommand{\thalf}{\frac{3}{2}}
\newcommand{\ddx}[1]{\frac{d^2#1}{dx^2}}
\newcommand{\ddr}[2]{\frac{\partial^2#1}{\partial#2^2}}
\newcommand{\mddr}[3]{\frac{\partial^2#1}{\partial#2\partial#3}}

\newcommand{\der}[2]{\frac{d#1}{d#2}}
\newcommand{\pder}[2]{\frac{\partial#1}{\partial#2}}
\newcommand{\half}{\frac{1}{2}}
\newcommand{\forth}{\frac{1}{4}}
\newcommand{\q}{\underline{q}}
\newcommand{\p}{\underline{p}}
\newcommand{\x}{\underline{x}}
\newcommand{\liu}{\hat{\mathcal{L}}}
\newcommand{\bigO}[1]{\mathcal{O}\left( #1 \right)}
\newcommand{\pauli}{\mathbf{\sigma}}
\newcommand{\bra}[1]{\langle#1|}
\newcommand{\ket}[1]{|#1\rangle}
\newcommand{\id}[1]{\mathbf{1}_{2^{#1}}}
\newcommand{\tinv}{\frac{1}{\tau}}
\newcommand{\s}{\sigma}
\newcommand{\vs}{\vec{\s}}
\newcommand{\vr}{\vec{r}}
\newcommand{\vq}{\vec{q}}
\newcommand{\vv}{\vec{v}}
\newcommand{\vo}{\vec{\omega}}
\newcommand{\uvs}{\underline{\vs}}
\newcommand{\expected}[1]{\left\langle #1 \right\rangle}
\newcommand{\D}{\Delta}

\newcommand{\range}[2]{#1, \ldots, #2}
\newcommand{\seq}[2]{#1 \ldots #2}
\newcommand{\psiCoef}[2]{\psi_{\range{#1}{#2}}}
\newcommand{\psiCoeff}[3]{\psi_{#1, \range{#2}{#3}}}
\newcommand{\mpa}[2]{#1^{(#2)}_{s_#2}}
\newcommand{\us}{\underline{s}}

\begin{document}

    \title{\sc\large Višje računske metode\\
		\bigskip
		\bf\Large Matrično produktni nastavki}
	\author{Mitja Vodnik, 28182041}
            \date{\today}
	\maketitle

    Bistvo te naloge je, da ustvarimo metodo za generiranje matrično produktnih nastavkov (MPA) za stanja verig spinov $\half$.
    To pomeni, da hočemo za neko stanje $\ket{\psi}$ verige dolžine $n$ generirati set matrik
    $\big\{\mpa{A}{j} | s_j \in \{0, 1\}, j = \range{1}{n}\big\}$, da velja:\\

    \begin{equation}\label{eq1}
        \ket{\psi} = \sum_{\range{s_1}{s_n}} \psiCoef{s_1}{s_n} \ket{\seq{s_1}{s_n}}
    \end{equation}

    \iffalse
    \begin{figure}
        \centering
        \includegraphics[width = \textwidth]{slika1.pdf}
        \caption{Odvisnost entropije prepletenosti simetrične biparticije osnovnega stanja od dolžine verige. Dolžine verig na tem grafu so le sode.}
        \label{slika1}
    \end{figure}

    \begin{figure}
        \centering
        \begin{subfigure}{\textwidth}
            \includegraphics[width = \textwidth]{slika4a.pdf}
        \end{subfigure}
        \begin{subfigure}{\textwidth}
            \includegraphics[width = \textwidth]{slika4b.pdf}
        \end{subfigure}
        \caption{Odvisnost entropije prepletenosti od velikosti blokov biparticije osnovnega stanja s periodičnimi robnimi pogoji.}
        \label{slika4}
    \end{figure}
    \fi

\end{document}
